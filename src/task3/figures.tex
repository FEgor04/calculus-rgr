\begin{frame}\frametitle{Влияение отображения на геометрические фигуры}
	Изучим, как меняются геометрические фигуры под действием отображения.
	Как и в прошлом пункте, будем строить фигуры в виртуальном пространстве,
	применять к точкам, лежащим на этих фигурах отображение и строить
	получившиеся точки в физическом пространстве.
\end{frame}

\begin{frame}{Отрезок \(u = v\), \(u \in [-10, 10]\)}
	Видно, что отрезок переходит в часть окружности, незамкнутую
	в окрестности точки \(w = 1\).
	При дальнейшем увеличении отрезка окрестность будет уменьшаться.
	\begin{figure}
		\centering
		\includegraphics[width=0.9\textwidth]{figures/conformal_u_equals_v.pdf}
	\end{figure}
\end{frame}

\begin{frame}{Парабола \(v = u^2\)}
	\begin{figure}
		\centering
		\includegraphics[width=0.9\textwidth]{figures/conformal_parabola.pdf}
		\caption{Парабола \(v=u^2\)}\label{fig:conformal_parabola}
	\end{figure}
\end{frame}

\begin{frame}{Окружность \((v-1)^2 + u^2 = 2\)}
	\begin{figure}
		\centering
		\includegraphics[width=0.9\textwidth]{figures/conformal_circle.pdf}
		\caption{Окружность \((v-1)^2 + u^2 = 2\)}\label{fig:conformal_circle}
	\end{figure}
\end{frame}
