\begin{frame}\frametitle{Обратное преобразование}
	Найдем для данного преобразования обратное.
	Для этого выразим \(z(w)\)
	\begin{align*}
		w(z) & = \frac{z-1}{z+1}                      \\
		z(w) & = \frac{1+w}{1-w} = 1 + \frac{2w}{1-w}
	\end{align*}

	Видно, что обратное преобразование конформно за исключением
	простого полюса \(w = 1\).
	Простым нулем обратного преобразования является точка \(w = -1\).

  Полюс \(w = 1\) и объясняет наличие выколотой точки \(w = 1\) 
  на предыщих графиках.
\end{frame}
