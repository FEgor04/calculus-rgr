\begin{frame}\frametitle{Задание 3. Конформные отображения}
\begin{equation*}
  w(z) = \frac{z-1}{z+1}
\end{equation*}

План выполнения работы:
\begin{enumerate}
  \item Рассмотреть конформное отображение. 
  Определить особые точки отображения (при наличии) и указать их вид.

  \item Изобразить на комплексной плоскости отображение
  области виртуального пространства в область физического пространства
  с помощью заданного преобразования. 

  \item Выделить действительную и мнимую части отображения
  для построения искривленной координатной сетки в физическом пространстве.

  \item Взять обратное преобразование к заданному и проанализировать его
  
  \item Расчитать профиль показателя преломления используя конформное отображение

\end{enumerate}
\end{frame}

% \subsection{Вывод по задаче}
\begin{frame}\frametitle{Вывод по задаче}
	\begin{itemize}
		\item Определили особые точки отображения
		\item Изобразили действие отображения на разные кривые
		\item Проанализировали обратное преобразование
		\item Рассчитали профиль показателя преобразования,
		      построили его график
	\end{itemize}
\end{frame}


