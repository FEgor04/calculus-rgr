\begin{frame}\frametitle{Задание 3. Конформные отображения}
	\begin{equation*}
		w(z) = \frac{z-1}{z+1} = 1 - \frac{2}{z+1}
	\end{equation*}

	План выполнения работы:
	\begin{enumerate}
		\item Рассмотреть конформное отображение.
		      Определить особые точки отображения (при наличии) и указать их вид.

		\item Изобразить на комплексной плоскости отображение
		      области виртуального пространства в область физического пространства
		      с помощью заданного преобразования.

		\item Выделить действительную и мнимую части отображения
		      для построения искривленной координатной сетки в физическом пространстве.

		\item Взять обратное преобразование к заданному и проанализировать его

		\item Рассчитать профиль показателя преломления используя конформное отображение

	\end{enumerate}
\end{frame}

% \subsection{Особые точки}
\begin{frame}
	\begin{figure}
		\centering
		\includegraphics[width=0.9\textwidth]{figures/conformal_map_1_plot.pdf}
		\caption{Прямая \(y=x\)}\label{fig:conformal_1}
	\end{figure}
\end{frame}

\begin{frame}
	\begin{figure}
		\centering
		\includegraphics[width=0.9\textwidth]{figures/conformal_map_2_plot.pdf}
		\caption{Прямая \(y = 5\)}\label{fig:conformal_2}
	\end{figure}
\end{frame}

\begin{frame}
	\begin{figure}
		\centering
		\includegraphics[width=0.9\textwidth]{figures/conformal_map_3_plot.pdf}
		\caption{Прямая \(x = 5\)}\label{fig:conformal_3}
	\end{figure}
\end{frame}



\begin{frame}
	\begin{figure}
		\centering
		\includegraphics[width=0.9\textwidth]{figures/conformal_map_4_plot.pdf}
		\caption{Парабола \(y=x^2\)}\label{fig:conformal_4}
	\end{figure}
\end{frame}

\begin{frame}
	\begin{figure}
		\centering
		\includegraphics[width=0.9\textwidth]{figures/conformal_map_5_plot.pdf}
		\caption{Гипербола \(y=1/x\)}\label{fig:conformal_5}
	\end{figure}
\end{frame}

% \subsection{Im(z) и Re(z)}
\begin{frame}{\(\Im w(z)\) и \(\Re w(z)\)}
	Для дальнейшего изучения отображения найдем \( \Im(w(z)) \) и \( \Re(w(z)) \).
	Пусть \( z = u + i v \). Тогда:

	\begin{align*}
		w(z) & = w(u + iv) = 1 - \frac{2}{(u + 1) + iv} = 1 - \frac{2((u + 1) - iv)}{((u+1)+iv)((u+1)-iv)} = \\
		     & = 1 - \frac{2(u + 1 - iv)}{(u+1)^2 + v^2} = 1 - \frac{2u + 2 - 2iv}{u^2 + 2u + v^2 + 1} =     \\
		     & = 1 - \frac{2u+2}{u^2+2u+v^2+1} - i\frac{2v}{u^2+2u+v^2+1} =                                  \\
		     & = \frac{u^2 +v^2 - 1}{u^2+2u+v^2+1} - i \frac{2v}{u^2+2u+v^2+1}
	\end{align*}

	Значит \( \Re(w(z)) = \frac{u^2+v^2-1}{u^2+2u+v^2+1} \), \( \Im(w(z)) = - \frac{2v}{u^2+2u+v^2+1} \)
\end{frame}

% \subsection{График v = u}
\begin{frame}
	\begin{figure}
		\centering
		\includegraphics[width=0.9\textwidth]{figures/conformal_map_1_plot.pdf}
		\caption{Прямая \(V=U\)}\label{fig:conformal_1}
	\end{figure}
\end{frame}

\begin{frame}
	\begin{figure}
		\centering
		\includegraphics[width=0.9\textwidth]{figures/conformal_map_2_plot.pdf}
		\caption{Прямая \(V = 5\)}\label{fig:conformal_2}
	\end{figure}
\end{frame}

\begin{frame}
	\begin{figure}
		\centering
		\includegraphics[width=0.9\textwidth]{figures/conformal_map_3_plot.pdf}
		\caption{Прямая \(U = 5\)}\label{fig:conformal_3}
	\end{figure}
\end{frame}



\begin{frame}
	\begin{figure}
		\centering
		\includegraphics[width=0.9\textwidth]{figures/conformal_map_4_plot.pdf}
		\caption{Парабола \(V=U^2\)}\label{fig:conformal_4}
	\end{figure}
\end{frame}

\begin{frame}
	\begin{figure}
		\centering
		\includegraphics[width=0.9\textwidth]{figures/conformal_map_5_plot.pdf}
		\caption{Гипербола \(V = 1/U\)}\label{fig:conformal_5}
	\end{figure}
\end{frame}


\begin{frame}
	\begin{figure}
		\centering
		\includegraphics[width=0.9\textwidth]{figures/conformal_map_6_plot.pdf}
		\caption{Полуокружность \(V = \sqrt{2^2-U^2}\)}\label{fig:conformal_6}
	\end{figure}
\end{frame}

% \subsection{Графики v = u^2}
\begin{frame}{Парабола \(v = u^2\)}
	\begin{figure}
		\centering
		\includegraphics[width=0.9\textwidth]{figures/conformal_parabola.pdf}
		\caption{Парабола \(v=u^2\)}\label{fig:conformal_parabola}
	\end{figure}
\end{frame}

\begin{frame}{Окружность \(v^2 + u^2 = (2 \sqrt 2)^2\)}
	\begin{figure}
		\centering
		\includegraphics[width=0.9\textwidth]{figures/conformal_circle.pdf}
    \caption{Окружность \(v^2 + u^2 = (2 \sqrt 2)^2\)}\label{fig:conformal_circle}
	\end{figure}
\end{frame}

% \subsection{Графики v = u^2}
\begin{frame}{Координатная сетка (горизонтальные прямые)}
	\begin{figure}
		\centering
		\includegraphics[width=0.9\textwidth]{figures/conformal_grid_horizontal.pdf}
		\caption{Координатная сетка (горизонтальные прямые)}\label{fig:conformal_grid_horizontal}
	\end{figure}
\end{frame}

\begin{frame}{Координатная сетка (вертикальные прямые)}
	\begin{figure}
		\centering
		\includegraphics[width=0.9\textwidth]{figures/conformal_grid_vertical.pdf}
		\caption{Координатная сетка (вертикальные прямые)}\label{fig:conformal_grid_vertical}
	\end{figure}
\end{frame}


\begin{frame}{Координатная сетка}
	\begin{figure}
		\centering
		\includegraphics[width=0.9\textwidth]{figures/conformal_grid_combined.pdf}
		\caption{Координатная сетка}\label{fig:conformal_grid}
	\end{figure}
\end{frame}

% \subsection{Обратное преобразование}
\begin{frame}\frametitle{Обратное преобразование}
	Найдем для данного преобразования обратное.
	Для этого выразим \(z(w)\)
	\begin{align*}
		w(z) & = \frac{z-1}{z+1}                      \\
		z(w) & = \frac{1+w}{1-w} = 1 + \frac{2w}{1-w}
	\end{align*}

	Видно, что обратное преобразование конформно за исключением
	простого полюса \(w = 1\).
	Простым нулем обратного преобразования является точка \(w = -1\).

  Полюс \(w = 1\) и объясняет наличие выколотой точки \(w = 1\) 
  на предыщих графиках.
\end{frame}

% \subsection{Показатель преломления}
\begin{frame}\frametitle{Профиль показателя преломления}
	Для расчета профиля показателя в физическом пространстве воспользуемся формулой:
	\begin{equation}
		n_z =
		\left|\frac{dw}{dz}\right| n_w =
		\frac{2}{(x+1)^2+y^2}
		\label{eq:pok_prelom}
	\end{equation}

\end{frame}

\begin{frame}\frametitle{Профиль показателя преломления}
	\begin{figure}
		\includegraphics[width=0.6\textwidth]{figures/prelom_plot.pdf}
		\caption{Профиль показателя преломления}
	\end{figure}
\end{frame}

% \subsection{Вывод по задаче}
\begin{frame}\frametitle{Вывод по задаче}
	\begin{itemize}
		\item Определили особые точки отображения
		\item Изобразили действие отображения на разные кривые
		\item Проанализировали обратное преобразование
		\item Рассчитали профиль показателя преобразования,
		      построили его график
	\end{itemize}
\end{frame}


