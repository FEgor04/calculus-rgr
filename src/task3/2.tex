\begin{frame}
	\begin{figure}
		\centering
		\includegraphics[width=0.9\textwidth]{figures/conformal_map_1_plot.pdf}
		\caption{Прямая \(V=U\)}\label{fig:conformal_1}
	\end{figure}
\end{frame}

\begin{frame}
	\begin{figure}
		\centering
		\includegraphics[width=0.9\textwidth]{figures/conformal_map_2_plot.pdf}
		\caption{Прямая \(V = 5\)}\label{fig:conformal_2}
	\end{figure}
\end{frame}

\begin{frame}
	\begin{figure}
		\centering
		\includegraphics[width=0.9\textwidth]{figures/conformal_map_3_plot.pdf}
		\caption{Прямая \(U = 5\)}\label{fig:conformal_3}
	\end{figure}
\end{frame}



\begin{frame}
	\begin{figure}
		\centering
		\includegraphics[width=0.9\textwidth]{figures/conformal_map_4_plot.pdf}
		\caption{Парабола \(V=U^2\)}\label{fig:conformal_4}
	\end{figure}
\end{frame}

\begin{frame}
	\begin{figure}
		\centering
		\includegraphics[width=0.9\textwidth]{figures/conformal_map_5_plot.pdf}
    \caption{Гипербола \(V = 1/U\)}\label{fig:conformal_5}
	\end{figure}
\end{frame}


\begin{frame}
	\begin{figure}
		\centering
		\includegraphics[width=0.9\textwidth]{figures/conformal_map_6_plot.pdf}
    \caption{Полуокружность \(V = \sqrt{2^2-U^2}\)}\label{fig:conformal_6}
	\end{figure}
\end{frame}
