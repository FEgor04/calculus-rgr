\begin{frame}\frametitle{Особые точки}
	Функция имеет две особые точки \(z_1 = 1\) и \(z_2 = -1\).
	Определим их вид.
	Для этого найдем производную \(w'(z)\).

	\[
		w'(z) = \frac{2}{(z+1)^2}
		\qquad
		w(z_1) = w(1) = 0
		\quad
		w'(z_1) = w'(1) \neq 0
	\]
	Значит точка \(z_1 = 1\) является простым нулем.
	Определим вид точки \(z_2 = -1\).

	\[ \lim_{z \to -1} \frac{z-1}{z+1} = \infty \]

	Для функции \(g(z) = 1/w(z) = \frac{z+1}{z-1}\)
	точка \(z_2 = -1\) является простым нулем.
	Значит точка \(z_2 = -1\) является для функции \(w(z)\) полюсом первого порядка.

\end{frame}
