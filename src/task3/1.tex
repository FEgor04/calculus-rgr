\begin{frame}\frametitle{Проверка}
Проверим, является ли данное отображение конформным.

\begin{itemize}
\item
\(w(z)\) аналитична при \(z \neq -1\).

\item
Для данного отображения существует обратное:
\begin{align*}
	w^{-1}(z) = \frac{1+w(z)}{1 - w(z)}
\end{align*}
Таким образом, \(w(z)\) -- биекция.

\item Найдем \(w'(z)\):
\begin{align*}
	w'(z) = \frac{2}{(z+1)^2} \neq 0 \: \forall z \in \Complex
\end{align*}
\end{itemize}

Таким образом, \(w(z)\) является конформным отображением
на \(\Complex \setminus \{1\}\).
\end{frame}

\begin{frame}\frametitle{Особые точки}
	Функция имеет две особые точки \(z_1 = 1\) и \(z_2 = -1\).
	Определим их вид.
	Для этого найдем производную \(w'(z)\).

	\[
		w'(z) = \frac{2}{(z+1)^2}
		\qquad
		w(z_1) = w(1) = 0
		\quad
		w'(z_1) = w'(1) \neq 0
	\]
	Значит точка \(z_1 = 1\) является простым нулем.
	Определим вид точки \(z_2 = -1\).

	\[ \lim_{z \to -1} \frac{z-1}{z+1} = \infty \]

	Для функции \(g(z) = 1/w(z) = \frac{z+1}{z-1}\)
	точка \(z_2 = -1\) является простым нулем.
	Значит точка \(z_2 = -1\) является для функции \(w(z)\) полюсом первого порядка.

\end{frame}

\begin{frame}{\(\Im w(z)\) и \(\Re w(z)\)}
  Для дальнейшего изучения отображения найдем \( \Im(w(z)) \) и \( \Re(w(z)) \).
  Пусть \( z = u + i v \). Тогда:

  \begin{align*}
    w(z) &= w(u + iv) = 1 - \frac{2}{(u + 1) + iv} = 1 - \frac{2((u + 1) - iv)}{((u+1)+iv)((u+1)-iv)} = \\
    &= 1 - \frac{2(u + 1 - iv)}{(u+1)^2 + v^2} = 1 - \frac{2u + 2 - 2iv}{u^2 + 2u + v^2 + 1} = \\
    &= 1 - \frac{2u+2}{u^2+2u+v^2+1} - i\frac{2v}{u^2+2u+v^2+1} = \\
    &= \frac{u^2 +v^2 - 1}{u^2+2u+v^2+1} - i \frac{2v}{u^2+2u+v^2+1}
  \end{align*}

  Значит \( \Re(w(z)) = \frac{u^2+v^2-1}{u^2+2u+v^2+1} \), \( \Im(w(z)) = - \frac{2v}{u^2+2u+v^2+1} \)
\end{frame}
