\begin{frame}\frametitle{Вычисление потока поля}
  Для нахождения потока поля
  \begin{equation*}
		\vec a = (\sin zy^2) \vec i + \sqrt{2} x \vec j + (\sqrt{2+y} -3z) \vec k
	\end{equation*}
		      через боковую поверхность тела \(T\), образованную вращением дуги \(AFEDC\)
		      вокруг оси \(Oy\), в направлении внешней нормали поверхности тела \(T\)
 

  Воспользумеся теоремой \textit{Остроградского-Гаусса}:
  \begin{equation*}
    \oiint\limits_{\Sigma}\left( \overrightarrow{\vec {a}}, \overrightarrow{\vec {n}} \right) d\sigma = \iiint\limits_V div \overrightarrow{\vec {a}} dxdydz 
  \end{equation*}
\end{frame}


\begin{frame}\frametitle{Вычисление потока поля}
  Найдем дивергент:
  \begin{equation*}
    div \overrightarrow{\vec a} = (\sin zy^2) + \sqrt{2} x + (\sqrt{2+y} - 3)
	\end{equation*}
	
  Поскольку $x^2 + z^2 = 1$ — это цилиндр, а $x^2 + y + z^2 = 2$ — параболоид, нам удобно перейти к цилиндрическим координатам:

  \begin{equation*}
    \begin{cases}
      x = r \cdot \cos(\theta) \\
      y = y \\
      z = r \cdot \sin(\theta)
    \end{cases}
  \end{equation*}
  
\end{frame}

\begin{frame}\frametitle{Вычисление потока поля} 

Тогда
  \begin{equation*}
    \oiint\limits_{\Sigma}\left( \overrightarrow{\vec {a}}, \overrightarrow{\vec {n}} \right) d\sigma = \iiint\limits_V (\sin zy^2) + \sqrt{2} x + (\sqrt{2+y} - 3) dV
  \end{equation*}

\end{frame}
