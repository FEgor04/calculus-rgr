\begin{frame}\frametitle{Вычисление потока поля}
	Для нахождения потока поля
	\begin{equation*}
		\vec a = (\sin zy^2) \vec i + \sqrt{2} x \vec j + (\sqrt{2+y} -3z) \vec k
	\end{equation*}
  через боковую поверхность тела \(T\), образованную вращением дуги \(AFEDC\)
	вокруг оси \(Oy\), в направлении внешней нормали поверхности тела \(T\)

	Воспользуемся теоремой \textit{Остроградского -- Гаусса}:
	\begin{equation*}
		\oiint\limits_{\Sigma}\left( \vec {a}, \vec {n} \right) \, d\sigma = \iiint\limits_V \Div \vec {a} \, dxdydz
	\end{equation*}
\end{frame}

\begin{frame}\frametitle{Вычисление потока поля}
	Найдем дивергенцию:
	\begin{equation*}
		\Div \vec a = \frac{\partial a_x}{\partial x} +  \frac{\partial a_y}{\partial y} +  \frac{\partial a_z}{\partial z} = 0 + 0 - 3 = -3
	\end{equation*}

	Поскольку $x^2 + z^2 = 1$ — это цилиндр, а $x^2 + y + z^2 = 2$ — параболоид, нам удобно перейти к цилиндрическим координатам:

	\begin{equation*}
		\begin{cases}
			x = r \cdot \cos \theta \\
			y = y                    \\
			z = r \cdot \sin \theta
		\end{cases}
	\end{equation*}

\end{frame}

\begin{frame}\frametitle{Вычисление потока поля} 
  Расставим пределы интегрирования:
  \begin{align*}
    r \in [0, 1], \
    \theta \in [0, 2\pi], \ 
    y = 2 - x^2 - z^2 = 2 - r^2
  \end{align*}

  Тогда
  \begin{equation*}
    \Phi_{\text{вращения}} = \oiint\limits_{\Sigma}\left( \vec {a}, \vec {n} \right) d\sigma = \iiint\limits_V -3 dV
  \end{equation*}
  
  \begin{equation*}
    = -3 \int\limits_{0}^{2 \pi} d \theta
    \int\limits_{0}^{1} r~dr 
    \int\limits_{0}^{2-r^2} dy 
    = -3 \int\limits_{0}^{2 \pi} d \theta
    \int\limits_{0}^{1} (2-r^2)r~dr 
  \end{equation*}
 
  \begin{equation*}
    = -3 \cdot 2 \pi \cdot
    (1 - \frac{1}{4})
    = - \frac{9}{2}\pi
  \end{equation*}

\end{frame}

\begin{frame}\frametitle{Вычисление потока поля}

  Расставим пределы интегрирования для конусовидного дна тела: 
  \begin{align*}
    r \in [0,1] \
    \theta \in [0, 2\pi] \ 
    y = -\sqrt{x^2 + z^2} = -\sqrt{r}
  \end{align*}

  Тогда
  \begin{equation*}
    \Phi_{\text{дна}}
    =  \oiint\limits_{D}\left( \vec {a}, \vec {n} \right) d\sigma = \iiint\limits_D -3 dD
  \end{equation*}
 
  \begin{equation*}
    = -3 \int\limits_{0}^{2 \pi} d \theta
    \int\limits_{0}^{1} r~dr 
    \int\limits_{-\sqrt{r}}^{0} dy 
    = -3 \int\limits_{0}^{2 \pi} d \theta
    \int\limits_{0}^{1}r^{\frac{3}{2}}~dr
    = -3 \cdot 2\pi \cdot \frac{2}{3} = -4 \pi
  \end{equation*}


  \begin{equation*}
    \Phi = \Phi_{\text{вращения}} - \Phi_{\text{дна}} = -\frac{9}{2}\pi - (-6 \pi) = \frac{3}{2}\pi
  \end{equation*}

\end{frame}
