\begin{frame}
\frametitle{Задание 2. Поток векторного поля}
  Дано тело \(T\), ограниченное следующими поверхностями:
\begin{equation*}
  z + \sqrt{4 - x^2 - y^2} = 0 \qquad x^2+z^2 = 1 \qquad x^2 + y + z^2 = 2
\end{equation*}
На рисунке предоставлено сечение тела \(T\) координатной плоскостью \(Oyz\).

\begin{itemize}
  \item Изобразите тело \(T\) на графике в пространстве.
  \item Вычислите поток поля
  \begin{equation*}
    \vec a = (\sin zy^2) \vec i + \sqrt{2} x \vec j + (\sqrt{2+y} -3k) \vec k
  \end{equation*}
  через боковую поверхность тела \(T\), образованную вращением дуги \(AFEDC\) 
  вокруг оси \(Oy\), в направлении внешней нормали поверхности тела \(T\).
\end{itemize}

\end{frame}
