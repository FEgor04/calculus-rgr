\begin{frame}\frametitle{Уравнения векторных линий}
% \begin{block}{Определение}
%   Векторной линией поля \(\vec A\) называется такая линия \(L\),
%   в каждой точке которой касательная к этой линии 
%   совпадает с направлением \(\vec A\).
% \end{block}

Для нахождения уравнений векторных линий решим дифференциальное уравнение:
  \begin{equation}
    \frac{dx}{e^x} = \frac{dy}{-e^y}
    \label{eq:vec_lines}
  \end{equation}
Проинтегрируем полученное уравнение:
  \begin{equation}
    \int e^{-x} \, dx = \int -e^{-y} \, dy
    \label{eq:vec_lines_int}
  \end{equation}
Интегрируя в уме, получаем:
  \begin{align*}
    -e^{-x} + C       &= e^{-y} + C \\
    e^{-y}  + e^{-x}   &= C
    \label{eq:vec_lines_integrated}
  \end{align*}

  Перенесем \(e^{-x}\) в правую часть и прологарифмируем:

  \begin{equation}
    y = \ln(C - e^{-x})
    \label{eq:vec_lines_final}
  \end{equation}

\end{frame}
