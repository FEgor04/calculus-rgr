\begin{frame}\frametitle{Потенциал векторного поля}
	Пусть \(U(R)\) -- потенциал поля \(\vec H\).

	\begin{equation}
		U(R) =
		\int\limits_{\widehat{AR}} \vec H \, d \vec r + C
		\label{eq:u_integral}
	\end{equation}


	\begin{columns}
		\begin{column}{0.5\textwidth}
			Возьмем в качестве $A$ точку $(0; 0)$.
			Так как интеграл (\ref{eq:u_integral}) не зависит от пути интегрирования,
			то разобьем его на две линии $(0; 0) - (x; 0) - (x; y)$.
		\end{column}
		\begin{column}{0.5\textwidth}
			\begin{figure}
				\centering
				\begin{tikzpicture}
					\draw[->, gray, very thin] (-2, 0) -- (2, 0);
					\draw[->, gray, very thin] (0, -2) -- (0, 2);

					\draw[black, thick] (0, 0) -- (1, 0);
					\draw[black, thick] (1, 0) -- (1, 1);
					\draw (0.5, 0.3) node {$x$};
					\draw (1.3, 0.45) node {$y$};
					\draw (1.2, 1.2) node {$R$};
					\draw (-0.3, -0.3) node {$0$};
				\end{tikzpicture}
				\caption{Путь интегрирования}
			\end{figure}
		\end{column}
	\end{columns}

\end{frame}

\begin{frame}\frametitle{Потенциал векторного поля: интегрирование}
	\begin{align*}
		U(x, y) & = \int\limits_{(0; 0)}^{(x; 0)} \left(e^x dx + (-e^y dy) \right) +
		\int\limits_{(x; 0)}^{(x; y)} \left(e^x dx + (-e^y dy) \right) + C =         \\
		        & =
		\int\limits_{0}^{x} e^x \, dx - \int\limits_{0}^{y} e^y \, dy + C =
		e^{x} - e^{y} + C
	\end{align*}
\end{frame}

\begin{frame}\frametitle{Потенциал векторного поля: проверка}
	По определению потенциала векторного поля~\cite[ст.~269]{zorich}, $\grad U = \vec H$. Проверим это.

	\begin{equation*}
		\grad U =
		\left( \frac{\partial U}{\partial x}; \frac{\partial U}{\partial y} \right) =
		\left( e^x; -e^y \right)
		= \vec H
	\end{equation*}

	Таким образом, \(U(x,y) = e^{x} - e^{y}\) -- потенциал векторного поля \(\vec H\).
\end{frame}
