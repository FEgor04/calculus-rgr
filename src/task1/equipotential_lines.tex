\begin{frame}\frametitle{Линии уровня потенциала}
	\begin{block}{Определение}
		Эквипотенциальная линия -- совокупность точек поля, имеющих один и тот же
		потенциал.
	\end{block}

	Для нахождения уравнения линий уровня потенциала зафиксируем
	уровень потенциала \(C\) и выразим \(y\) через \(x\)

	\begin{equation*}
		U(x, y) = e^x - e^y = C \Longleftrightarrow e^y = e^x - C
	\end{equation*}
	Прологарифмируем уравнение с обеих сторон и получим
	\begin{equation}
		y = \ln(e^x - C) \qquad e^x > C \Rightarrow x > \ln(C)
		\label{eq:potential_lines}
	\end{equation}

\end{frame}

\begin{frame}\frametitle{Линии уровня потенциала}
	На рис.~\ref{fig:potential_lines} представлены графики функций
	(\ref{eq:vec_lines}) черным цветом и (\ref{eq:potential_lines})
	разными цветами для $C = e^1, e^2\, \ldots, e^9$
	\begin{figure}
		\centering
		\includegraphics[width=0.5\textwidth]{figures/potential_lines_plot.pdf}
		\caption{Линии уровня потенциала поля \(\vec H\)}\label{fig:potential_lines}
	\end{figure}

\end{frame}
