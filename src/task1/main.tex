\begin{frame}\frametitle{Задание 1. Потенциал векторного поля}
  Дано векторное поле \( \vec H = \left( e^x; -e^y \right) \).

План:
\begin{itemize}
  \item Убедитесь, что поле потенциально
  \item Найдите уравнения векторных линий
  \item Изобразите векторные линии на рисунке
  \item Изобразите линии уровня потенциала (эквипотенциальные линии).
  Проиллюстрируйте ортогональность линий уровня и векторных линий.
  \item Зафиксируйте точки \( A \) и \( B \) на какой-либо векторной линии.
  Вычислите работу поля вдоль этой линии.
\end{itemize}
\end{frame}

% \subsection{Потенциальность поля}
\begin{frame}\frametitle{Потенциальность поля}
Убедимся, что поле потенциально.
Воспользуемся критерием потенциальности векторного поля.

\begin{block}{Критерий потенциальности векторного поля}
  Непрерывное в области \( D \subset \mathbb{R}^n \) векторное поле \(\vec A\)
  потенциально в \(D\) тогда и только тогда, когда его работа на любом лежащем в \(D\)
  замкнутом пути равна нулю:
  \begin{equation*}
    \oint \vec H \cdot d \vec s = 0
  \end{equation*}
\end{block}

Проверим этот критерий для \( \vec H = (e^x, - e^y) \).
Очевидно, что векторное поле \(H\) непрерывно на \(\mathbb{R}^2\).

\end{frame}   

% \subsection{Уравнения векторных линий}
\begin{frame}\frametitle{Уравнения векторных линий}
	% \begin{block}{Определение}
	%   Векторной линией поля \(\vec A\) называется такая линия \(L\),
	%   в каждой точке которой касательная к этой линии 
	%   совпадает с направлением \(\vec A\).
	% \end{block}

	Для нахождения уравнений векторных линий решим дифференциальное уравнение:
	\begin{equation}
		\frac{dx}{e^x} = \frac{dy}{-e^y}
		\label{eq:vec_lines}
	\end{equation}
	Проинтегрируем полученное уравнение:
	\begin{equation}
		\int e^{-x} \, dx = \int -e^{-y} \, dy
		\label{eq:vec_lines_int}
	\end{equation}
	Интегрируя в уме, получаем:
	\begin{align*}
		-e^{-x} + C_1    & = e^{-y} + C_2 \\
		e^{-y}  + e^{-x} & = C
		\label{eq:vec_lines_integrated}
	\end{align*}

	Перенесем \(e^{-x}\) в правую часть и прологарифмируем:

	\begin{equation}
		y = - \ln(C - e^{-x}), \qquad
		C - e^{-x} > 0 \Longleftrightarrow x > - \ln(C), C > 0
		\label{eq:vec_lines_final}
	\end{equation}

\end{frame}

% \subsection{Рисунок векторных линий}
\begin{frame}\frametitle{Векторные линии}
	% make figures/vec_lines.pdf
	На рис.~\ref{fig:vec_lines} черным цветом нарисованы векторные линии (\ref{eq:vec_lines})
	для $C \in \{e^1, e^2, \ldots, e^9\}$, синим - векторное поле в данных точках.
	\begin{figure}
		\centering
		\includegraphics[width=0.5\textwidth]{figures/vec_lines_plot.pdf}
		\caption{Векторные линии поля \(\vec H\)}\label{fig:vec_lines}
	\end{figure}

\end{frame}

% \subsection{Линии уровня потенциала}
\begin{frame}\frametitle{Потенциал векторного поля}
	Пусть \(U(R)\) -- потенциал поля \(\vec H\).

	\begin{equation*}
		U(R) =
		\int\limits_{\widehat{AR}} \vec H \, d \vec r + C
		\label{eq:u_integral}
	\end{equation*}

	Возьмем в качестве $A$ точку $(0; 0)$.
	Так как интеграл в уравнении (\ref{eq:u_integral}) не зависит от пути,
	то разобьем его на две линии $(0; 0) - (x; 0) - (x; y)$

	\begin{align*}
		U(x, y) & = \int\limits_{(0; 0)}^{(x; 0)} \left(e^x dx + (-e^y dy) \right) +
		\int\limits_{(x; 0)}^{(x; y)} \left(e^x dx + (-e^y dy) \right) + C =         \\
		        & =
		\int\limits_{0}^{x} e^x \, dx - \int\limits_{0}^{y} e^y \, dy + C =
		e^{x} - e^{y} + C
	\end{align*}

\end{frame}

\begin{frame}\frametitle{Потенциал векторного поля: проверка}
	По определению потенциала векторного поля~\cite[ст.~269]{zorich}, $\grad U = \vec H$. Проверим это.

	\begin{equation*}
		\grad U =
		\left( \frac{\partial U}{\partial x}; \frac{\partial U}{\partial y} \right) =
		\left( e^x; -e^y \right)
		= \vec H
	\end{equation*}

	Таким образом, \(U(x,y) = e^{x} - e^{y}\) -- потенциал векторного поля \(\vec H\).
\end{frame}

% \subsection{Работа поля вдоль линии}
\begin{frame}\frametitle{Линии уровня потенциала}
  \begin{block}{Определение}
    Эквипотенциальная линия -- совокупность точек поля, имеющих один и тот же
    потенциал.
  \end{block}

  Для нахождения уравнения линий уровня потенциала зафиксируем
  уровень потенциала \(C\) и выразим \(y\) через \(x\)

  \begin{equation}
    U(x, y) = e^x - e^y = C \Longleftrightarrow e^y = e^x - C
    \label{eq:potential_level_line}
  \end{equation}
  Прологарифмируем уравнение с обеих сторон и получим
  \begin{equation}
    y = \ln(e^x - C) \qquad e^x > C \Rightarrow x > \ln(C)
  \end{equation}
  
\end{frame}

\begin{frame}\frametitle{Линии уровня потенциала}
  \begin{figure}
    \centering
		\includegraphics[width=0.5\textwidth]{figures/potential_lines_plot.pdf}
  \end{figure}
  
\end{frame}

% \subsection{Вывод по задаче}
\begin{frame}\frametitle{Вывод по задаче}
	\begin{itemize}
		\item Установили, что \(\vec H\) --- потенциально
		\item Нашли уравнения векторных линий
		\item Нашли потенциал поля
		\item Нашли уравнение линий уровня потенциала
		\item Изобразили векторные линии и линии уровня потенциала графически
		\item Нашли работу поля вдоль векторной линии
	\end{itemize}
\end{frame}

