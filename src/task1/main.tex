\begin{frame}\frametitle{Задание 1. Потенциал векторного поля}
  Дано векторное поле \( \vec H = \left( e^x; -e^y \right) \).

План:
\begin{itemize}
  \item Убедитесь, что поле потенциально
  \item Найдите уравнения векторных линий
  \item Изобразите векторные линии на рисунке
  \item Изобразите линии уровня потенциала (эквипотенциальные линии).
  Проиллюстрируйте ортогональность линий уровня и векторных линий.
  \item Зафиксируйте точки \( A \) и \( B \) на какой-либо векторной линии.
  Вычислите работу поля вдоль этой линии.
\end{itemize}
\end{frame}

% \subsection{Потенциальность поля}
\begin{frame}\frametitle{Потенциальность поля}
	Убедимся, что поле потенциально.
	Для этого найдем \( \rot \vec H = \grad \vec H \times \vec H \).

	\begin{equation*}
		\grad \vec H =
		\left( \frac{\partial \vec H}{\partial x};  \frac{\partial \vec H}{\partial y}\right) =
		\left( e^x; -e^{y} \right)
	\end{equation*}
	\begin{align*}
		\rot \vec H = \grad \vec H \times \vec H & = (e^x; -e^y; 0) \times (e^x; -e^y; 0) =               \\
		                                         & = (0, 0, e^x \cdot (-e^y) - (-e^y) \cdot e^x) = \vec 0
	\end{align*}
	Таким образом, так как \(\rot \vec H = \vec 0\), поле \(\vec H\) -- потенциально~\cite{korn}.

\end{frame}

% \subsection{Уравнения векторных линий}
\begin{frame}\frametitle{Уравнения векторных линий}
	% \begin{block}{Определение}
	%   Векторной линией поля \(\vec A\) называется такая линия \(L\),
	%   в каждой точке которой касательная к этой линии 
	%   совпадает с направлением \(\vec A\).
	% \end{block}

	Для нахождения уравнений векторных линий решим дифференциальное уравнение:
	\begin{equation}
		\frac{dx}{e^x} = \frac{dy}{-e^y}
		\label{eq:vec_lines}
	\end{equation}
	Проинтегрируем полученное уравнение:
	\begin{equation}
		\int e^{-x} \, dx = \int -e^{-y} \, dy
		\label{eq:vec_lines_int}
	\end{equation}
	Интегрируя в уме, получаем:
	\begin{align*}
		-e^{-x} + C_1    & = e^{-y} + C_2 \\
		e^{-y}  + e^{-x} & = C
		\label{eq:vec_lines_integrated}
	\end{align*}

	Перенесем \(e^{-x}\) в правую часть и прологарифмируем:

	\begin{equation}
		y = - \ln(C - e^{-x}), \qquad
		C - e^{-x} > 0 \Longleftrightarrow x > - \ln(C), C > 0
		\label{eq:vec_lines_final}
	\end{equation}

\end{frame}

% \subsection{Рисунок векторных линий}
\begin{frame}\frametitle{Векторные линии}
	% make figures/vec_lines.pdf
  На рис.~\ref{fig:vec_lines} черным цветом нарисованы векторные линии (\ref{eq:vec_lines})
  для $C \in \{e^1, e^2, \ldots, e^9\}$, синим - векторное поле в данных точках.
	\begin{figure}
		\centering
		\includegraphics[width=0.5\textwidth]{figures/vec_lines_plot.pdf}
		\caption{Векторные линии поля \(\vec H\)}\label{fig:vec_lines}
	\end{figure}

\end{frame}

% \subsection{Вывод по задаче}
\begin{frame}\frametitle{Вывод по задаче}
	\begin{itemize}
		\item Установили, что \(\vec H\) --- потенциально
		\item Нашли уравнения векторных линий
		\item Нашли потенциал поля
		\item Нашли уравнение линий уровня потенциала
		\item Изобразили векторные линии и линии уровня потенциала графическиo
		\item Нашли работу поля вдоль векторной линии
	\end{itemize}
\end{frame}

