\begin{frame}\frametitle{Задание 1. Потенциал векторного поля}
	Дано векторное поле \( \vec H = \left( e^x; -e^y \right) \).

	План:
	\begin{enumerate}
		\item Убедитесь, что поле потенциально
		\item Найдите уравнения векторных линий
		\item Изобразите векторные линии на рисунке
		\item Найдите потенциал поля при помощи криволинейного интеграла
		\item Изобразите линии уровня потенциала (эквипотенциальные линии).
		      Проиллюстрируйте ортогональность линий уровня и векторных линий.
		\item Зафиксируйте точки \( A \) и \( B \) на какой-либо векторной линии.
		      Вычислите работу поля вдоль этой линии.
	\end{enumerate}
\end{frame}

\subsection{Потенциальность поля}
\begin{frame}\frametitle{Потенциальность поля}
	\begin{block}{Необходимое условие потенциальности поля}
		Пусть \(\vec H\) -- векторное поле.
		Тогда, если в некотором шаре выполняется условие
		\(\frac{\partial H_x}{\partial y} = \frac{\partial H_y}{\partial x}\),
		то поле \(\vec H\) потенциально в этом шаре~\cite[ст.~270,272]{zorich}.
	\end{block}

	\[
		\frac{\partial H_x}{\partial y} = \frac{\partial (e^x)}{\partial y} = 0
		\qquad
		\frac{\partial H_y}{\partial x} = \frac{\partial (-e^y)}{\partial x} = 0
	\]

	Очевидно, что необходимое условие выполняется на \(\Real^2\), а значит
	поле \(\vec H\) потенциально на \(\Real^2\).

\end{frame}

\subsection{Уравнение векторных линий}
\begin{frame}\frametitle{Уравнения векторных линий}
	Для нахождения уравнений векторных линий решим дифференциальное уравнение:
	\begin{equation*}
		\frac{dx}{e^x} = \frac{dy}{-e^y}
		\label{eq:vec_lines_definition}
	\end{equation*}
	Проинтегрируем полученное уравнение:
	\begin{equation*}
		\int e^{-x} \, dx = \int -e^{-y} \, dy
		\label{eq:vec_lines_int}
	\end{equation*}
	Интегрируя, получаем:
	\begin{align*}
		-e^{-x} + C_1    & = e^{-y} + C_2 \\
		e^{-y}  + e^{-x} & = C
		\label{eq:vec_lines_integrated}
	\end{align*}

	Перенесем \(e^{-x}\) в правую часть и прологарифмируем:

	\begin{equation}
		y = - \ln(C - e^{-x}), \qquad
		C - e^{-x} > 0 \Longleftrightarrow x > - \ln(C), C > 0
		\label{eq:vec_lines}
	\end{equation}

\end{frame}

\subsection{Рисунок векторных линий}
\begin{frame}\frametitle{Векторные линии}
	На рис.~\ref{fig:vec_lines} черным цветом нарисованы векторные линии (\ref{eq:vec_lines})
	для $C \in \{e^1, e^2, \ldots, e^9\}$, синим - векторное поле в данных точках.
	\begin{figure}
		\centering
		\includegraphics[width=0.5\textwidth]{figures/vec_lines_plot.pdf}
		\caption{Векторные линии поля \(\vec H\)}\label{fig:vec_lines}
	\end{figure}

\end{frame}

\subsection{Потенциал векторного поля}
\begin{frame}\frametitle{Потенциал векторного поля}
	Пусть \(U(R)\) -- потенциал поля \(\vec H\).

	\begin{equation}
		U(R) =
		\int\limits_{\widehat{AR}} \vec H \, d \vec r + C
		\label{eq:u_integral}
	\end{equation}


	\begin{columns}
		\begin{column}{0.5\textwidth}
			Возьмем в качестве $A$ точку $(0; 0)$.
			Так как интеграл (\ref{eq:u_integral}) не зависит от пути интегрирования,
			то разобьем его на две линии $(0; 0) - (x; 0) - (x; y)$.

			Путь интегрирования показан на рисунке~\ref{fig:integration_path}.
		\end{column}
		\begin{column}{0.5\textwidth}
			\begin{figure}
				\centering
				\begin{tikzpicture}
					\draw[->, gray, very thin] (-2, 0) -- (2, 0);
					\draw[->, gray, very thin] (0, -2) -- (0, 2);

					\draw[black, thick] (0, 0) -- (1, 0);
					\draw[black, thick] (1, 0) -- (1, 1);
					\draw (1.2, 1.2) node {$(x, y)$};
					\draw (1.2, -0.3) node {$(x, 0)$};
					\draw (-0.3, -0.3) node {$0$};
				\end{tikzpicture}
				\caption{Путь интегрирования}\label{fig:integration_path}
			\end{figure}
		\end{column}
	\end{columns}

\end{frame}

\begin{frame}\frametitle{Потенциал векторного поля: интегрирование}
	\begin{align*}
		U(x, y) & = \int\limits_{(0; 0)}^{(x; 0)} \left(e^x dx + (-e^y dy) \right) +
		\int\limits_{(x; 0)}^{(x; y)} \left(e^x dx + (-e^y dy) \right) + C =         \\
		        & =
		\int\limits_{0}^{x} e^x \, dx - \int\limits_{0}^{y} e^y \, dy + C =
		e^{x} - e^{y} + C
	\end{align*}
\end{frame}

\begin{frame}\frametitle{Потенциал векторного поля: проверка}
	По определению потенциала векторного поля~\cite[ст.~269]{zorich}, $\grad U = \vec H$. Проверим это.

	\begin{equation*}
		\grad U =
		\left( \frac{\partial U}{\partial x}; \frac{\partial U}{\partial y} \right) =
		\left( e^x; -e^y \right)
		= \vec H
	\end{equation*}

	Таким образом, \(U(x,y) = e^{x} - e^{y}\) -- потенциал векторного поля \(\vec H\).
\end{frame}

\subsection{Линии уровня потенциала}
\begin{frame}\frametitle{Линии уровня потенциала}
	\begin{block}{Определение}
		Эквипотенциальная линия -- совокупность точек поля, имеющих один и тот же
		потенциал.
	\end{block}

	Для нахождения уравнения линий уровня потенциала зафиксируем
	уровень потенциала \(C\) и выразим \(y\) через \(x\)

	\begin{equation*}
		U(x, y) = e^x - e^y = C \Longleftrightarrow e^y = e^x - C
	\end{equation*}
	Прологарифмируем уравнение с обеих сторон и получим
	\begin{equation}
		y = \ln(e^x - C) \qquad e^x > C \Rightarrow x > \ln(C)
		\label{eq:potential_lines}
	\end{equation}

\end{frame}

\begin{frame}\frametitle{Линии уровня потенциала}
	На рис.~\ref{fig:potential_lines} представлены графики функций
	(\ref{eq:vec_lines}) черным цветом и (\ref{eq:potential_lines})
	разными цветами для $C = e^1, e^2\, \ldots, e^9$
	\begin{figure}
		\centering
		\includegraphics[width=0.5\textwidth]{figures/potential_lines_plot.pdf}
		\caption{Линии уровня потенциала поля \(\vec H\)}\label{fig:potential_lines}
	\end{figure}

\end{frame}

\subsection{Работа поля вдоль линии}
\begin{frame}\frametitle{Работа поля вдоль линии}
	Зафиксируем точки \(A = (-5; 10)\) и \(B = (5; -5)\).
	Найдем работу поля $\vec H$ вдоль векторной линии,
	проходящей через эти точки.

	Тогда работа поля вдоль линии будет равна:
	\begin{align*}
		\int\limits_{AB} \vec H \, d \vec s
		 & =
		U(B) - U(A)
		=
		(e^{B_x} - e^{B_y}) - (e^{A_x} - e^{A_y}) = \\
		 & = (e^5 - e^{-5}) - (e^{-5} - e^{10}) =
		e^5 - 2 e^{-5} + e^{10} =                   \\
		 & \approx 22174.86548
		\label{eq:work_across_line}
	\end{align*}
\end{frame}

\subsection{Вывод по задаче}
\begin{frame}\frametitle{Вывод по задаче}
	\begin{itemize}
		\item Установили, что \(\vec H\) --- потенциально
		\item Нашли уравнения векторных линий
		\item Нашли потенциал поля
		\item Нашли уравнение линий уровня потенциала
		\item Изобразили векторные линии и линии уровня потенциала графическиo
		\item Нашли работу поля вдоль векторной линии
	\end{itemize}
\end{frame}

