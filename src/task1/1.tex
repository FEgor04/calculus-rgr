\begin{frame}\frametitle{Потенциальность поля}
Убедимся, что поле потенциально.
Для этого найдем \( \rot \vec H = \grad \vec H \times \vec H \).

\begin{equation*}
  \grad \vec H = 
  \left( \frac{\partial \vec H}{\partial x};  \frac{\partial \vec H}{\partial y}\right) =
  \left( e^x; -e^{y} \right)
\end{equation*}
\begin{align*}
  \rot \vec H = \grad \vec H \times \vec H &= (e^x; -e^y; 0) \times (e^x; -e^y; 0) = \\
                                           &= (0, 0, e^x \cdot (-e^y) - (-e^y) \cdot e^x) = \vec 0
\end{align*}
Таким образом, так как \(\rot \vec H = \vec 0\), поле \(\vec H\) -- потенциально~\cite{korn}.

\end{frame}   
