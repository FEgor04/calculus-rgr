\begin{frame}\frametitle{Задача 1. Потенциал векторного поля}
Убедимся, что поле потенциально.
Воспользуемся критерием потенциальности векторного поля.

\begin{block}{Критерий потенциальности векторного поля}
  Непрерывное в области \( D \subset \mathbb{R}^n \) векторное поле \(\vec A\)
  потенциально в \(D\) тогда и только тогда, когда его работа на любом лежащем в \(D\)
  замкнутом пути равна нулю:
  \begin{equation*}
    \oint \vec H \cdot d \vec s = 0
  \end{equation*}
\end{block}

Проверим этот критерий для \( \vec H = (e^x, - e^y) \).
Очевидно, что векторное поле \(H\) непрерывно на \(\mathbb{R}^2\).

\end{frame}   
