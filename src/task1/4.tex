\begin{frame}\frametitle{Потенциал векторного поля}
  Пусть \(U(R)\) -- потенциал поля \(\vec H\).
  
  \begin{equation}
      U(R) =
      \int\limits_{\widehat{AR}} \vec H \, d \vec r
    \label{eq:u_integral}
  \end{equation}
  Где $A$ -- точка поля, координаты которой удовлетворяют условиям
  существования полей $\vec H$ и $\rot \vec H$.

  Возьмем в качестве $A$ точку $(0; 0)$.
  Так как интеграл в уравнении (\ref{eq:u_integral}) не зависит от пути,
  то разобьем его на две линии $(0; 0) - (R_x: 0) - (R_x; R_y)$

  \begin{align}
    U(R) &= \int\limits_{(0; 0)}^{(R_x; 0)} \left(e^x dx + (-e^y dy) \right) +
    \int\limits_{(R_x; 0)}^{(R_x; R_y)} \left(e^x dx + (-e^y dy) \right) = \\
         &=
    \int\limits_{0}^{R_x} e^x \, dx - \int\limits_{0}^{R_y} e^y \, dy =
    e^{R_x} - e^{R_y}
    \label{eq:}
  \end{align}
	
\end{frame}

\begin{frame}\frametitle{Поетнциал векторного поля: проверка}
  По определению потенциала векторного поля, $\grad U = \vec H$. Проверим это.

  \begin{equation}
    \grad U =
    \left( \frac{\partial U}{\partial x}; \frac{\partial U}{\partial y} \right) =
    \left( e^x; -e^y \right)
    = \vec H
  \end{equation}

  Таким образом, \(U(R) = e^{R_x} - e^{R_y}\) -- потенциал векторного поля \(\vec H\).
\end{frame}
