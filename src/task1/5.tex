\begin{frame}\frametitle{Линии уровня потенциала}
  \begin{block}{Определение}
    Эквипотенциальная линия -- совокупность точек поля, имеющих один и тот же
    потенциал.
  \end{block}

  Для нахождения уравнения линий уровня потенциала зафиксируем
  уровень потенциала \(C\) и выразим \(y\) через \(x\)

  \begin{equation}
    U(x, y) = e^x - e^y = C \Longleftrightarrow e^y = e^x - C
    \label{eq:potential_level_line}
  \end{equation}
  Прологарифмируем уравнение с обеих сторон и получим
  \begin{equation}
    y = \ln(e^x - C) \qquad e^x > C \Rightarrow x > \ln(C)
  \end{equation}
  
\end{frame}

\begin{frame}\frametitle{Линии уровня потенциала}
  \begin{figure}
    \centering
		\includegraphics[width=0.5\textwidth]{figures/potential_lines_plot.pdf}
  \end{figure}
  
\end{frame}
